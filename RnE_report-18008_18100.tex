%%!TeX program = xelatex
%% 부득이하게 pdflatex을 사용해야 할 경우 위의 magic comment를 제거하십시오.

% Initiated by 정민석(2014년도 경기과학고 수학과전문교원)
% Continously being modified by 경기과학고 TeX 사용자협회
% Website : http://gshslatexintro.github.io 

\documentclass{gshs_report}
% 아래의 함수를 사용하면 이미지 파일들을 같은 디렉토리 내에 images 라는 이름을 가진 폴더를 생성한 후, 그 폴더 안에 넣어 사용할 수 있습니다.
% 사용하고자 한다면 주석을 푸십시오.
\graphicspath{{images/}}
% 이곳에 필요한 별도의 패키지들을 적어넣으시오.
%\usepackage{...}
\usepackage{verbatim} % for commment, verbatim environment
\usepackage{spverbatim} % automatic linebreak verbatim environment
%\usepacakge{indentfirst}
\usepackage{tikz}
%\tikzset{
%	image label/.style={
%		every node/.style={
			%fill=black,
			%text=white,
%			font=\sffamily\scriptsize,
%			anchor=south west,
%			xshift=0,
%			yshift=0,
%			at={(0,0)}
%		}
%	}
%}
\usepackage{amsmath}
\usepackage{amsfonts}
\usepackage{amssymb}
\usepackage{float}
\usepackage{graphicx}
\usepackage{tabularx}
\usepackage{multirow}
\usepackage{booktabs}
\usepackage{longtable}
\usepackage{gensymb}
%\usepackage{subcaption}
%\usepackage{floatrow}
%\usepackage{pict2e}
%\usepackage[backend=biber,style=authoryear]{biblatex}
%\usepackage{biblatex}
\usepackage{pgfplots}
\pgfplotsset{
	compat=newest,
	label style={font=\sffamily\scriptsize},
	ticklabel style={font=\sffamily\scriptsize},
	legend style={font=\sffamily\tiny},
	major tick length=0.1cm,
	minor tick length=0.05cm,
	every x tick/.style={black},
}

\usetikzlibrary{shapes}
\usetikzlibrary{plotmarks}
\usepackage{listings}
\usepackage{hologo}
\usepackage{makecell}

\lstset{
	basicstyle=\small\ttfamily,
	columns=flexible,
	breaklines=true
}

\citation

\bibdata

%: ----------------------------------------------------------------------
%:               보고서 정보를 입력하시오
% ----------------------------------------------------------------------
% 아래와 같은 command를 만들면 길이가 긴 용어를 간편하게 사용할 수 있습니다. 단, 이미 지정된 함수명들은 새로운 함수명으로 사용할 수 없습니다.
\newcommand{\gshs}{Gyeonggi Science High School for the Gifted }

\researchtype{기초} % 기초 / 심화
\reporttype{결과} % 중간 / 결과

\title{보고서 제목} % 제목 개행 시 \linebreak 사용. \\나 \newline 은 안됨.
\englishtitle{English title}% 제목 개행 시 \linebreak 사용. \\나 \newline 은 안됨.

\author[1] {홍길동} % 제 1 저자명
\email[1]{hong@e-mail.address} % 제 1 저자 이메일
\author[2] {전우치} % 제 2 저자명
\email[2]{cheon@e-mail.address} % 제 2 저자 이메일
\advisor{박기현} % 지도교사명
\advisorEmail{guitar79@gs.hs.kr} % 지도교사 이메일

%%%%%%%%%%%%%%%%%%%%%%%%%%%%%%%%%%%%%%%%%%%%%%%
%%%% researchtype이 '심화'일 경우에만 나타남 %%%%
\professor{교수님} % 지도교수명
\professorEmail{professor@e-mail.address} % 지도교수 이메일
%%%%%%%%%%%%%%%%%%%%%%%%%%%%%%%%%%%%%%%%%%%%%%%%
\summitdate{2018}{02}{07} % 제출일 (연, 월, 일)
\newtheorem{definition}{정의}
 % usepackage 등의 명령어는 여기에.
\usepackage{cite}
\usepackage{textcomp}
\usepackage{tocloft}
\setlength{\cftbeforesecskip}{0pt}
\setlength{\cftbeforesubsecskip}{0pt}
\setlength{\cftbeforesubsubsecskip}{0pt}

% 본문 시작
\begin{document}

%표지만들기
%makecover 함수와 관련하여 "Underfull \hbox (badness 10000) in paragraph" 오류는 무시하십시오. (TeXstudio ver 2.9.4 오류 기준)
%\makecover

%초록(영문)
%\maketitle  % command to print the title page with above variables
\makecover  % command to print the title page with above variables

\setcounter{page}{1}
\renewcommand{\thepage}{\roman{page}}

%----------------------------------------------
%   Table of Contents (자동 작성됨)
%----------------------------------------------
\cleardoublepage
\addcontentsline{toc}{section}{Contents}
\setcounter{secnumdepth}{3} % organisational level that receives a numbers
\setcounter{tocdepth}{3}    % print table of contents for level 3
\baselineskip=2.2em
\tableofcontents


%----------------------------------------------
%     List of Figures/Tables (자동 작성됨)
%----------------------------------------------
\cleardoublepage
\clearpage
\listoffigures	% 그림 목록과 캡션을 출력한다. 만약 논문에 그림이 없다면 이 줄의 맨 앞에 %기호를 넣어서 코멘트 처리한다.

\cleardoublepage
\clearpage
\listoftables  % 표 목록과 캡션을 출력한다. 만약 논문에 표가 없다면 이 줄의 맨 앞에 %기호를 넣어서 코멘트 처리한다.


\cleardoublepage
\clearpage

%---------------------------------------------------------------------
%                  영문 초록을 입력하시오
%---------------------------------------------------------------------
%\begin{abstracts}     %this creates the heading for the abstract page
%	\addcontentsline{toc}{section}{Abstract}  %%% TOC에 표시
%	\noindent{
%			Put your abstract here. Once upon a time, \gshs said : `The first, and the best.'
%	}
%\end{abstracts}

%\cleardoublepage
%\clearpage

\begin{abstractskor}
	\addcontentsline{toc}{section}{초록}  %%% TOC에 표시
	\noindent{
본 논문에서는 초점 조절 구동 시스템을 구현하는 방법을 제안한다. 초점 조절 구동 펌웨어는 기본적으로 Arduino를 사용하며 여러 기능이 존재하여 자유로운 설정이 가능하다. ASCOM 드라이버는 C\# 코딩을 이용하여 컴퓨터로 정보 전달이 가능하다. 본 논문에서 제안된 방법은 사람이 손으로 제어하는 것보다 정밀하고 빠르게 천체망원경의 초점을 맞출 수 있도록 편의성을 제공하기 위한 바탕 역할을 할 수 있다.
	}
\end{abstractskor}






 % Abstract

%%%%%%%%%%%%%%%%%%%%%%%%%%%%%%%%%%%%%%%%%%%%%%%%%%%%%%%%%%%
%%%% Main Document %%%%%%%%%%%%%%%%%%%%%%%%%%%%%%%%%%%%%%%%
%%%%%%%%%%%%%%%%%%%%%%%%%%%%%%%%%%%%%%%%%%%%%%%%%%%%%%%%%%%
\cleardoublepage
\clearpage
\renewcommand{\thepage}{\arabic{page}}
\setcounter{page}{1}

%각 장을 아래와 같이 sub 폴더 안에 만들어서 넣으면 편리하다.
\section{분석}

\subsection{identification}

test


\section{결과}

이후에 나오는 Countour map에서 Orion A Cloud의 경우 10'가 실제 길이로는 1.25pc, $\rho$ Ophiuchus Cloud에서는 10'가 실제 길이로는 0.40pc이다. 그리고 Line profile에서 검은선은 $^{12}CO$ 천이 선, 초록선은 $^{13}CO$ 천이 선, 갈색선은 $C^{18}O$ 천이 선의 line이며 파랑색과 빨강색 점선은 blue, red lobe의 속도 범위를 나타낸 것이다.


\subsection{Orion A Cloud}



\clearpage
\newpage   
FIR2의 contour map을 살펴보면 N-S 방향으로 강한 방출류가 보인다. 방출류의 크기가 약 30 arcsec정도로 다른 방출류들보다 훨씬 작다. 본 연구의 결과가 Takahashi보다 약 3배 더 약하게 나타났다. Aso는 이 천체에 대하여 분석을 하지 않았다. SED의 기울기 $\alpha$는 3.12으로 Class I으로 분류하였다. Furlan에서도 Class I으로 분류하였다.\cite{HerschelFurlan} 

FIR3의 contour map을 살펴보면  red lobe와 blue lobe의 중심이 거의 같은 위치에 있다. 방출류가 거의 시선방향과 나란하다는 것을 알 수 있다. 본 연구의 결과가 Takahashi보다 약 20배 더 약하게 나타났다. Takahashi는 red와 blue lobe를 각각 2개씩 관측했다. Aso는 이 천체에 대하여 분석을 하지 않았다.  SED의 기울기 $\alpha$는 1.51으로 Class I으로 분류하였다. Furlan에서도 Class I으로 분류하였다.\cite{HerschelFurlan} 

FIR6b의 contour map을 살펴보면 주변의 다른 별들로 인해서 방출류 구조 말고 다른 별들에서 나온 방출류로 인한 선들이 많이 보인다. NW-SE 방향으로 방출류가 관측이 된다. 본 연구의 결과가 Takahashi보다 약 4배 더 약하게 나타났다. Aso는 이 천체에 대하여 분석을 하지 않았다. SED의 기울기 $\alpha$는 1.20으로 Class I으로 분류하였다. Furlan에서는 Class 0으로 분류하였다.\cite{HerschelFurlan} 

MMS2의 contour map을 살펴보면 red lobe와 blue lobe가 둘 다 별을 기준으로 동쪽에 있는 특이한 모양을 하고 있다. SW 방향에 보이는 방출류 구조는 MMS5로 인한 방출류이다. 아마 이것에 의해 방출류가 영향을 받아 치우쳐졌을 가능성이 있다. 본 연구의 결과가 Takahashi보다 약 4배 더 약하게 나타났다. J=1-0 을 사용한 Aso보다 약 1.8배 더 강하게 나타났다. Aso는 MMS2, MMS3, MMS4 세 개의 원시성을 하나의 원시성으로 간주하고 방출류를 계산했다.\cite{Aso} SED의 기울기 $\alpha$는 1.23으로 Class I으로 분류하였다. Furlan에서는 Flat으로 분류하였다.\cite{HerschelFurlan} 

MMS5의 contour map을 살펴보면 E-W 방향으로 방출류가 관측이 된다. blue lobe가 red lobe보다 더 강하게 관측된다. 본 연구의 결과가 Takahashi보다 약 4배 더 약하게 나타났다. Aso보다 약 10\% 강하게 나타났다.\cite{Aso} SED의 기울기 $\alpha$가 3.17로 Class I로 분류되어야 하지만 $2.2\mu m$와 $20 \mu m$ 사이의 관측 데이터의 값이 $10^{-15}$ Wm$^{-2}$정도로 매우 작게 나타났기 때문에 Class 0으로 분류하였다. Furlan에서는 Class I으로 분류하였다.\cite{HerschelFurlan} 

MMS9의 contour map을 살펴보면  E-W 방향으로 강한 방출류가 나오는 것을 볼 수 있다. Takahashi(2008)에서는 red lobe를 두개 관측했는데, 본 연구의 관측 자료에서도 blue lobe의 중심 부근에서 작은 red lobe가 존재하는것을 알 수 있다. 그 세기는 main red lobe보다 10배 정도 더 작은것으로 관측되었다. 본 연구의 결과가 Takahashi보다 약 20배 더 약하게 나타났다. Takahashi는 2개의 red lobe를 관측했다.  Aso의 결과에 비해서는 약 6.7배 약하게 나타났다.\cite{Aso} SED의 기울기 $\alpha$는 1.53으로 Class I으로 분류하였다. Furlan에서는 Class 0으로 분류하였다.\cite{HerschelFurlan} 
\\

각 원시성들의 line profile에 나타난 $^{13}CO$와 $C^{18}O$ 천이 선을 보면 red peak, center, blue peak 세 지점에서 모두 비슷한 개형이 나타났다. 따라서 각 원시성 주변의 red, blue lobe가 본 연구에서 관찰한 원시성으로부터 나온 방출류라는 것을 알 수 있다. 그리고 SED로부터 분류한 진화단계가 일부 원시성들은 Furlan의 결과와 다르게 나타났는데, Furlan은 각 원시성들을 bolometric temperature을 기준으로 분류했기 때문에 본 연구에서 SED를 이용해서 구한 classification과 차이가 있을 수 있다.

\clearpage
\newpage   


\subsection{방출류의 세기}
\begin{table}[h]
	\begin{center}
		\begin{tabular}{c|c|c|c}
			\toprule
			\textbf{Name} &$\mathbf{F_{R}}$ & $\mathbf{F_{B}}$ & $\mathbf{F_{CO}}$\\
			& \multicolumn{3}{c}{[M$_{\odot}$ km s$^{-1}$ yr$^{-1}$]}\\
			\midrule
			\multicolumn{4}{c}{Orion A Cloud}\\
			\midrule
			FIR2 & 1.14E-05 & 3.28E-05 & 4.42E-05\\
			FIR3 & 4.77E-03 & 7.43E-03 & 1.22E-04\\
			FIR6b & 1.13E-05 & 1.18E-05 & 2.31E-05\\
			MMS2 & 1.14E-05 & 4.50E-05 & 5.64E-05\\
			MMS5 & 5.80E-06 & 1.55E-05 & 2.13E-05\\
			MMS9 & 3.67E-06 & 1.09E-05 & 1.46E-05\\
			\midrule
			\multicolumn{4}{c}{$\rho$ Ophiuchus Cloud}\\
			\midrule
			Elias 32 & 1.77E-06 & 1.01E-05 & 1.19E-05\\
			IRS 46 & 4.56E-07 & 7.14E-07 & 1.17E-06\\
			VLA 1623 & 2.42E-06 & 3.15E-06 & 5.57E-06\\
			BBRCG 24 & 3.78E-07 & 8.19E-07 & 1.20E-06\\
		\end{tabular}
	\end{center}
	\caption{관측된 원시성들의 방출류의 세기}
\end{table}

표에서 $F_R$와 $F_B$는 각각 red, blue lobe의 방출류의 세기를 구한것이다.(\ref{FCO}) $F_{CO}$는 두 값을 더한 값으로 원시성이 방출해내는 총 방출류의 세기이다. 두 영역의 방출류를 비교해 보면 $\rho$ Ophiuchus Cloud은 Orion A Cloud보다 질량이 작고 광도가 낮은 별들이 탄생하는 영역으로 방출류의 세기 또한 작게 나타났다.\\
 % body1
%\section{분석}

\subsection{identification}

test


\section{결과}

이후에 나오는 Countour map에서 Orion A Cloud의 경우 10'가 실제 길이로는 1.25pc, $\rho$ Ophiuchus Cloud에서는 10'가 실제 길이로는 0.40pc이다. 그리고 Line profile에서 검은선은 $^{12}CO$ 천이 선, 초록선은 $^{13}CO$ 천이 선, 갈색선은 $C^{18}O$ 천이 선의 line이며 파랑색과 빨강색 점선은 blue, red lobe의 속도 범위를 나타낸 것이다.


\subsection{Orion A Cloud}


FIR2의 contour map을 살펴보면 N-S 방향으로 강한 방출류가 보인다. 방출류의 크기가 약 30 arcsec정도로 다른 방출류들보다 훨씬 작다. 본 연구의 결과가 Takahashi보다 약 3배 더 약하게 나타났다. Aso는 이 천체에 대하여 분석을 하지 않았다.
각 원시성들의 line profile에 나타난 $^{13}CO$와 $C^{18}O$ 천이 선을 보면 red peak, center, blue peak 세 지점에서 모두 비슷한 개형이 나타났다. 따라서 각 원시성 주변의 red, blue lobe가 본 연구에서 관찰한 원시성으로부터 나온 방출류라는 것을 알 수 있다. 그리고 SED로부터 분류한 진화단계가 일부 원시성들은 Furlan의 결과와 다르게 나타났는데, Furlan은 각 원시성들을 bolometric temperature을 기준으로 분류했기 때문에 본 연구에서 SED를 이용해서 구한 classification과 차이가 있을 수 있다. \cite{bontemps1996evolution}


\subsection{방출류의 세기}
\begin{table}[h]
	\begin{center}
		\begin{tabular}{c|c|c|c}
			\toprule
			\textbf{Name} &$\mathbf{F_{R}}$ & $\mathbf{F_{B}}$ & $\mathbf{F_{CO}}$\\
			& \multicolumn{3}{c}{[M$_{\odot}$ km s$^{-1}$ yr$^{-1}$]}\\
			\midrule
			\multicolumn{4}{c}{Orion A Cloud}\\
			\midrule
			FIR2 & 1.14E-05 & 3.28E-05 & 4.42E-05\\
			FIR3 & 4.77E-03 & 7.43E-03 & 1.22E-04\\
			FIR6b & 1.13E-05 & 1.18E-05 & 2.31E-05\\
			MMS2 & 1.14E-05 & 4.50E-05 & 5.64E-05\\
			MMS5 & 5.80E-06 & 1.55E-05 & 2.13E-05\\
			MMS9 & 3.67E-06 & 1.09E-05 & 1.46E-05\\
			\midrule
			\multicolumn{4}{c}{$\rho$ Ophiuchus Cloud}\\
			\midrule
			Elias 32 & 1.77E-06 & 1.01E-05 & 1.19E-05\\
			IRS 46 & 4.56E-07 & 7.14E-07 & 1.17E-06\\
			VLA 1623 & 2.42E-06 & 3.15E-06 & 5.57E-06\\
			BBRCG 24 & 3.78E-07 & 8.19E-07 & 1.20E-06\\
		\end{tabular}
	\end{center}
	\caption{관측된 원시성들의 방출류의 세기}
\end{table}

 Ophiuchus Cloud은 Orion A Cloud보다 질량이 작고 광도가 낮은 별들이 탄생하는 영역으로 방출류의 세기 또한 작게 나타났다. \cite{megeath2012spitzer}\\
 % 원래 있던 문

\bibliography{bibfile} % 참고문헌
% BibTeX 코드 쉽게 얻어오는 방법 %
% Google Scholar 에서 검색한 결과에서 `인용'을 클릭한다.
% BibTeX 코드를 얻고자 한다면, 하단의 `BibTeX' 을 클릭.
% 코드가 나온다. Ctrl+A, Ctrl+C로 복사, bibfile에 붙여넣기.


\end{document}
