\section{서론}

\subsection{연구의 필요성 및 목적}

\subsubsection{연구의 필요성}

천체를 관측할 때 초점을 맞춘다면 관측할 천체의 모습이 더 선명하게 보인다. 일반적으로 대부분의 망원경은 초점을 손으로 맞출 수 있게 설계되어있다. 하지만 모터포커서가 있다면 손으로 초점을 맞추는 것보다 정확하게 초점을 맞출 수 있게 된다. 모터 포커서와 천체 사진 분석 기술을 활용하면 사람이 손으로 초점을 맞추는 것보다 정확하게 초점을 찾을 수 있을 것이다. 실제로 모터 포커서와 연계해 초점을 맞춰주는 소프트웨어도 몇 종류가 있으나 오류가 발생하는 경우가 있다. 따라서 천체망원경의 모터 포커서의 컨트롤러 구동 시스템을 개발하면 여러 천체를 관측하는 데 있어서 보다 정확한 사진들을 얻을 수 있을 것이다.
그림 1, 그림 2에서 알 수 있듯이 모터포커싱을 이용하여 초점을 맞추면 아무것도 하지 않고 그냥 관측했을 때에 비해서 훨씬 정확하게 천체를 관측할 수 있게 된다. 그림 1과 그림 \ref{fig:9}를 비교하여 보면 그림 2의 표면이 훨씬 더 선명하다는 사실을 알 수 있다.
\begin{figure}
	\includegraphics[width=\linewidth]{9.bmp}
	\caption{그림 캡션입니다.}
	\label{fig:9}
\end{figure}

\subsubsection{연구 목적}

이 연구는 아두이노를 이용하여 천체망원경을 이용한 천체관측을 시행할 때 필요한 모터포커서를 조정할 수 있는 모터포커서 컨트롤러 구동 시스템을 구현하는 것을 첫 번째 목표로 한다. 만약 이를 성공하게 되면, 초점을 자동으로 조정하는 알고리즘을 만들어서 천체의 초점을 자동으로 맞출 수 있도록 할 것이다.

\subsection{이론적 배경}

\subsubsection{Micro touch와 모터의 조절}

그림 3이 바로 Micro touch로, 시중에 나와있는 모터포커서이다. 이를 옆의 컴퓨터와 연결시킨 그림이 바로 그림4로, 이를 이용하여 컴퓨터에서도 ASCOM이라는 프로그램을 이용하여 원격으로 모터의 초점을 
맞출 수 있도록 설정할 수가 있다. 그림3에서 나온 위의 두 버튼(IN, OUT)은 각각 초점을 맞추기 위해 망원경의 길이를 줄이거나 늘일 수 있는 버튼이다. 
Micro touch를 수동 혹은 자동으로 작동시켜 IN또는 OUT의 명령을 내렸을 경우, 그림 5에 보이는 모터포커서가 작동하게 된다. 이 모터포커서는 그림 5의 오른쪽에 보이는 모터를 움직여 천체망원경의 경통의 길이를 조절할 수 있도록 한다. 경통의 길이가 변화하면 그에 따라서 빛이 퍼지는 정도가 달라지므로 이를 잘 조정하면 망원경으로 관측하는 천체의 초점을 맞출 수 있게 된다.

자동사진 분류를 연구한 논문을 참고하였다\cite{haningu2009}.