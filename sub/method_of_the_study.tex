\section{연구과정 및 방법}



\subsection{천체망원경 모터포커서 컨트롤러 구동 시스템 개발}



\subsubsection{하드웨어 제작}


1. STM32L432KC STM32L432KC는 아두이노 nano 보다 성능이 좋은 arduino로, 방향만 반대일 뿐 핀의 순서와 종류가 모두 동일하여 우리가 이후에 arduino nano로 회로도 제작 후 더 좋은 성능을 위하여 이를 사용하였다. STM32L432KC와 아두이노가 다른점은 STM32L432KC는 3.6V를 사용한다는 점이다.
2.SSD1306 SSD1306은 OLED의 한 종류로, 컴퓨터에 있는 Serial Monitor가 아닌 OLED기판 위에 내용이 출력되도록 한다. 크기는 128*64이고, 전압은 3V 또는 5V를 사용한다.
3.DHT22 DHT22는 아두이노에서 사용가능한 센서로, 주위의 온도와 습도를 측정하여 Serial Monitor 또는 OLED에 출력을 해준다. DHT22의 전압은 3.3V~5.5V이다.
4.DRV8825 DRV8825는 stepper motor controller로, 


\subsubsection{아두이노의 활용}



아두이노는 마이크로컨트롤러를 달고 있는 기판으로, 아두이노의 여러 가지 핀에 전선을 연결한 뒤에 코딩을 하여 아두이노에 업로드를 하면 아두이노가 코딩된 내용을 그대로 실행할 수 있도록 하는 하드웨어이다. 본 논문에서 밝힌 모터포커서는 이 아두이노를 이용해서 만들 것이다. 아두이노가 작동하기 위해서는 5V의 전원이 공급되어야 하는 데, 모터를 돌리기 위해서는 약 12V가 필요하므로 이 전력을 이용하여 아두이노를 같이 실행하게 될 것이다.



\subsubsection{아두이노를 이용한 모터포커서 제작}



아두이노를 이용하여 모터포커서를 만드는 데 필요한 기술들은 크게 3가지이다. 먼저, 아두이노를 이용하여 모터를 돌릴 수 있어야 하는데, 본 논문에서 돌려야 할 모터의 종류는 ‘스테핑모터’이다. 스테핑모터는 여러 모터 중에서도 정밀한 각도또는 위치를 제어해야 하는 경우에 사용하는 모터이다. 두 번째로, 스위치를 활용하여 모터의 움직임을 조절할 수 있어야 한다. 마지막으로, 모터를 조절한 것을 노트북에 연결되어있지 않아도 상황을 볼 수 있도록 모터가 얼마나 돌아가 있는지, 혹은 이로 인해 늘어난 경통의 길이가 얼마나 되는지 확인할 수 있도록 아두이노를 이용하여 LED판을 실행시켜 진행 상황을 확인할 수 있어야 한다. 각 기술을 실현하는 코드와 기판 모양은 fritzing과 같은 프로그램으로 저장하여 활용할 수 있도록 한다.

이 세가지를 모두 아두이노로 실행시킬 수 있게 되면, 납땜하거나 PCB 기판으로 만들어서 모터 포커서를 완성한다. 완성된 모터포 커서가 잘 작동하는지 실제 천체망원경을 이용하여 실험하여본다.



\subsection{자동초점조절 알고리즘 구현}



자동초점조절 알고리즘은 모터포커서를 이용할 때, 모터포커서 뿐만이 아니라 모터포커서를 이용하여 초점이 움직인 사진을 카메라로 분석하여 초점이 잘 맞았는지 확인한 뒤에 피드백 작용으로 초점이 맞을 때까지 스스로 초점을 조절하는 장치이다. 이 알고리즘에서 가장 중요한 것들은 카메라로 찍은 사진을 초점이 더 잘 맞았는지 분석할 수 있어야 하며, 초점을 조절할 때 맞는 초점을 넘어가 버려서 끝없이 초점이 가운데에서 왔다 갔다 하는 일이 없도록 해야 한다는 점이다.

결국, 자동초점조절 알고리즘은

1. 원래 사진을 찍는다.

2. 입력된 사진을 이용하여 실험으로 IN 또는 OUT을 실행시킨다.

3. 2에서 초점이 잘 맞았는지 잘 안 맞았는지 판단하고, 모터를 돌려야 할 방향을 설정한다. 

4. 모터를 돌려 초점을 맞춘다.

5. 초점이 잘 맞았는지 확인한다.

의 순서로 진행될 수 있도록 설계되어야 할 것이다.

