\section{결론}

\subsection{모터 초점 조절 장치 컨트롤러 및 펌웨어}

스테핑 모터를 이용하여 직접 설정한 최소 스텝 단위로 모터가 돌아갈 수 있도록 컨트롤러를 개발하였다. 또한, Arduino 코딩을 기반으로 기존의 초점 조절 장치보다 더 다양한 새로운 기능들을 추가하여 개발하였다. 이러한 초점 조절 장치가 이용된다면 현재보다 더욱 편리하게 천체망원경의 초점을 맞출 수 있게 되어 천체를 관측하여 사진을 찍을 때 사람의 손보다 수월하게 진행할 수 있게 만들기 위한 기반이 되어줄 수 있다.

\subsection{무선통신 개발}

블루투스와 WIFI module을 이용하여 직접 천체망원경으로 가지 않더라고 초점을 조절할 수 있도록 해주는 프로그램을 개발하여 천체망원경과 가까운 거리에서 필요하다면 초점을 맞출 수 있는 기능, 즉 사람들의 편의성을 증대시켰다.

\subsection{ASCOM 드라이버 개발}

ASCOM 드라이버를 Protocol에 맞게 개발하여 이후에 할 수 있는 추후 연구와의 정보 전달 역할, 즉 매개체 역할을 할 수 있게 하였다.
