\section{결론}

\subsection{아두이노를 이용한 스테핑모터의 회전}

그림 9-1과 그림 9-2를 참조하면, 12V의 전원과 100uF의 콘덴서를 이용하면 delay만큼의 간격을 두고 일정한 각도로 회전하는 스테핑 모터를 관측할 수 있었다. 이를 기본으로 하여 스위치와 oled를 적절히 조합하기만 하면 우리가 원할 때 움직이는 모커포커서를 구현할 수 있을 것으로 보인다.

\subsection{자동초점조절 알고리즘 구현}

자동초점조절 알고리즘은 먼저 모커포커서가 만들어져 있어야지 구현을 할 수 있다. 모터포커서를 이용하여 초점을 맞추는 과정을 거칠 수 있기 때문이다. 모터포커서를 완성한 뒤에 자동초점 알고리즘에 대해 더 연구해 볼 것이다.
