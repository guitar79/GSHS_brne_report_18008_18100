\section{서론}

\subsection{연구의 필요성 및 목적}

\subsubsection{연구의 필요성}

천체를 관측할 때 초점을 맞춘다면 관측할 천체의 모습이 더 선명하게 보인다. 일반적으로 대부분의 망원경은 초점을 손으로 맞출 수 있게 설계되어있다. 하지만 모터포커서가 있다면 손으로 초점을 맞추는 것보다 정확하게 초점을 맞출 수 있게 된다. 모터 포커서와 천체 사진 분석 기술을 활용하면 사람이 손으로 초점을 맞추는 것보다 정확하게 초점을 찾을 수 있을 것이다. 실제로 모터 포커서와 연계해 초점을 맞춰주는 소프트웨어도 몇 종류가 있으나 오류가 발생하는 경우가 있다. 따라서 천체망원경의 모터 포커서의 컨트롤러 구동 시스템을 개발하면 여러 천체를 관측하는 데 있어서 보다 정확한 사진들을 얻을 수 있을 것이다.
그림 1, 그림 2에서 알 수 있듯이 모터포커싱을 이용하여 초점을 맞추면 아무것도 하지 않고 그냥 관측했을 때에 비해서 훨씬 정확하게 천체를 관측할 수 있게 된다. 그림 1과 그림 2를 비교하여 보면 그림 2의 표면이 훨씬 더 선명하다는 사실을 알 수 있다.

\subsubsection{연구 목적}

이 연구는 아두이노를 이용하여 천체망원경을 이용한 천체관측을 시행할 때 필요한 모터포커서를 조정할 수 있는 모터포커서 컨트롤러 구동 시스템을 구현하는 것을 첫 번째 목표로 한다. 만약 이를 성공하게 되면, 초점을 자동으로 조정하는 알고리즘을 만들어서 천체의 초점을 자동으로 맞출 수 있도록 할 것이다.

\subsection{이론적 배경}

\subsubsection{Micro touch와 모터의 조절}

그림 3이 바로 Micro touch로, 시중에 나와있는 모터포커서이다. 이를 옆의 컴퓨터와 연결시킨 그림이 바로 그림4로, 이를 이용하여 컴퓨터에서도 ASCOM이라는 프로그램을 이용하여 원격으로 모터의 초점을 
맞출 수 있도록 설정할 수가 있다. 그림3에서 나온 위의 두 버튼(IN, OUT)은 각각 초점을 맞추기 위해 망원경의 길이를 줄이거나 늘일 수 있는 버튼이다. 
Micro touch를 수동 혹은 자동으로 작동시켜 IN또는 OUT의 명령을 내렸을 경우, 그림 5에 보이는 모터포커서가 작동하게 된다. 이 모터포커서는 그림 5의 오른쪽에 보이는 모터를 움직여 천체망원경의 경통의 길이를 조절할 수 있도록 한다. 경통의 길이가 변화하면 그에 따라서 빛이 퍼지는 정도가 달라지므로 이를 잘 조정하면 망원경으로 관측하는 천체의 초점을 맞출 수 있게 된다.

\section{연구과정 및 방법}

\subsection{천체망원경 모터포커서 컨트롤러 구동 시스템 개발}

\subsubsection{아두이노의 활용}

 아두이노는 마이크로컨트롤러를 달고 있는 기판으로, 아두이노의 여러 가지 핀에 전선을 연결한 뒤에 코딩을 하여 아두이노에 업로드를 하면 아두이노가 코딩된 내용을 그대로 실행할 수 있도록 하는 하드웨어이다. 본 논문에서 밝힌 모터포커서는 이 아두이노를 이용해서 만들 것이다. 아두이노가 작동하기 위해서는 5V의 전원이 공급되어야 하는 데, 모터를 돌리기 위해서는 약 12V가 필요하므로 이 전력을 이용하여 아두이노를 같이 실행하게 될 것이다.
 
 \subsubsection{아두이노를 이용한 모터포커서 제작}
 
  아두이노를 이용하여 모터포커서를 만드는 데 필요한 기술들은 크게 3가지이다. 먼저, 아두이노를 이용하여 모터를 돌릴 수 있어야 하는데, 본 논문에서 돌려야 할 모터의 종류는 ‘스테핑모터’이다. 스테핑모터는 여러 모터 중에서도 정밀한 각도또는 위치를 제어해야 하는 경우에 사용하는 모터이다. 두 번째로, 스위치를 활용하여 모터의 움직임을 조절할 수 있어야 한다. 마지막으로, 모터를 조절한 것을 노트북에 연결되어있지 않아도 상황을 볼 수 있도록 모터가 얼마나 돌아가 있는지, 혹은 이로 인해 늘어난 경통의 길이가 얼마나 되는지 확인할 수 있도록 아두이노를 이용하여 LED판을 실행시켜 진행 상황을 확인할 수 있어야 한다. 각 기술을 실현하는 코드와 기판 모양은 fritzing과 같은 프로그램으로 저장하여 활용할 수 있도록 한다.
 이 세가지를 모두 아두이노로 실행시킬 수 있게 되면, 납땜하거나 PCB 기판으로 만들어서 모터 포커서를 완성한다. 완성된 모터포 커서가 잘 작동하는지 실제 천체망원경을 이용하여 실험하여본다.
 
 \subsection{자동초점조절 알고리즘 구현}
 
  자동초점조절 알고리즘은 모터포커서를 이용할 때, 모터포커서 뿐만이 아니라 모터포커서를 이용하여 초점이 움직인 사진을 카메라로 분석하여 초점이 잘 맞았는지 확인한 뒤에 피드백 작용으로 초점이 맞을 때까지 스스로 초점을 조절하는 장치이다. 이 알고리즘에서 가장 중요한 것들은 카메라로 찍은 사진을 초점이 더 잘 맞았는지 분석할 수 있어야 하며, 초점을 조절할 때 맞는 초점을 넘어가 버려서 끝없이 초점이 가운데에서 왔다 갔다 하는 일이 없도록 해야 한다는 점이다.
 결국, 자동초점조절 알고리즘은
 1. 원래 사진을 찍는다.
 2. 입력된 사진을 이용하여 실험으로 IN 또는 OUT을 실행시킨다.
 3. 2에서 초점이 잘 맞았는지 잘 안 맞았는지 판단하고, 모터를 돌려야 할 방향을 설정한다. 
 4. 모터를 돌려 초점을 맞춘다.
 5. 초점이 잘 맞았는지 확인한다.
 의 순서로 진행될 수 있도록 설계되어야 할 것이다.
 
\section{결과 및 토의}

\subsection{천체망원경 모터 포커서 컨트롤러 구동 시스템 개발}

\subsubsection{온습도 센서를 이용한 온도 및 습도의 측정}

 이 연구를 진행하는 데 아두이노를 사용하는 것이 가장 기본이라고 판단하였기 때문에 아두이노로 실행할 수 있는 것들 중 쉬운 축이라고 생각되는 온습도 센서(dht22)를 활용하여 온습도를 측정하는 일이었다. 그림6과 같이 기판을 짜고 코드를 입력하면 serial 모니터에 온도와 습도가 delay만큼의 간격을 두고 계속 출력된다.
이를 응용하여 그림 7처럼 oled(oled1306)에 온도와 습도를 실시간으로 출력하는 프로그램을 만들 수도 있다.

\subsubsection{스테핑모터의 회전}

스테핑 모터의 종류는 2가지가 있다. 하나는 전선이 6개가 연결된 것과 전선이 4개가 연결되어 있는 것이다. 우리가 찾은 코드는 전선이 4개만 연결하는 것에 대한
▲그림 8. DRV8825의 구조
코드였기 때문에, 구멍인 6개인 것을 4개인 것에 대응시켰다. 전선이 6개인 것의 1번, 3번, 4번, 6번을 연결하면 구멍이 4개인 것과 같은 효과를 낼 수 있다. (순서가 반대로 된다면 모터의 회전 방향이 반대가 될 것이다) 또한, 모터으 회전 방향을 제어하기 위해서는 아두이노에 모터드라이버를 사용하여야 한다. 모터드라이버는 drv8825를 사용한다. 스테핑 모터에 있어서 이 모터드라이버가 있으면 움직임을 더욱 정밀하게 설정할 수 있는데, 아두이노의 초기 설정에서 M0, M1, M2(MODE)의 값이 1이냐 0이냐에 따라 풀스텝(1.8)에서부터 1/32스텝까지 한번 실행할 때마다 회전시킬 수 있다. (delay에 따라회전 속도를 조절할 수 있다.)(표1 참조)

\subsection{자동초점조절}
	
	자동초점조절 알고리즘은 모터포커서를 다 제작한 뒤에 구현하여도 문제가 없을 것으로 판단하여 자동초점조절 알고리즘 구현은 모터포커서를 다 제작할 뒤에 연구하도록 할 것이다.
	
	\section{결론}
	
	\subsection{아두이노를 이용한 스테핑모터의 회전}
	
	그림 9-1과 그림 9-2를 참조하면, 12V의 전원과 100uF의 콘덴서를 이용하면 delay만큼의 간격을 두고 일정한 각도로 회전하는 스테핑 모터를 관측할 수 있었다. 이를 기본으로 하여 스위치와 oled를 적절히 조합하기만 하면 우리가 원할 때 움직이는 모커포커서를 구현할 수 있을 것으로 보인다.
	
	\subsection{자동초점조절 알고리즘 구현}
	
	자동초점조절 알고리즘은 먼저 모커포커서가 만들어져 있어야지 구현을 할 수 있다. 모터포커서를 이용하여 초점을 맞추는 과정을 거칠 수 있기 때문이다. 모터포커서를 완성한 뒤에 자동초점 알고리즘에 대해 더 연구해 볼 것이다.
	
	\section{참고문헌}
	
	  [1] 한인구, 김형국. "자동 개인 사진 분류장치." 정보 및 제어 논문집,  (2009.10): 105-106.
	[2] 이덕규, 육영춘, 연정흠, 장수영, 이응식. "고해상도 전자광학카메라 초점조절장치 개발." 한국항공우주학회 학술발표회 초록집,  (2014.11): 553-555.
	[3] 한헌수, 최정렬. "실시간 감시 카메라를 구현하기 위한 고속 영상확대 및 초점조절 기법." 한국정밀공학회지, 21.3 (2004.3): 74-82.
	[4] DRV8825,Website:https://www.pololu.com/product/2131--drv8825

 \cite{megeath2012spitzer}\\
