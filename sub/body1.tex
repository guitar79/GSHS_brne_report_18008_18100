\section{분석}

\subsection{identification}

test


\section{결과}

이후에 나오는 Countour map에서 Orion A Cloud의 경우 10'가 실제 길이로는 1.25pc, $\rho$ Ophiuchus Cloud에서는 10'가 실제 길이로는 0.40pc이다. 그리고 Line profile에서 검은선은 $^{12}CO$ 천이 선, 초록선은 $^{13}CO$ 천이 선, 갈색선은 $C^{18}O$ 천이 선의 line이며 파랑색과 빨강색 점선은 blue, red lobe의 속도 범위를 나타낸 것이다.


\subsection{Orion A Cloud}



\clearpage
\newpage   
FIR2의 contour map을 살펴보면 N-S 방향으로 강한 방출류가 보인다. 방출류의 크기가 약 30 arcsec정도로 다른 방출류들보다 훨씬 작다. 본 연구의 결과가 Takahashi보다 약 3배 더 약하게 나타났다. Aso는 이 천체에 대하여 분석을 하지 않았다. SED의 기울기 $\alpha$는 3.12으로 Class I으로 분류하였다. Furlan에서도 Class I으로 분류하였다.\cite{HerschelFurlan} 

FIR3의 contour map을 살펴보면  red lobe와 blue lobe의 중심이 거의 같은 위치에 있다. 방출류가 거의 시선방향과 나란하다는 것을 알 수 있다. 본 연구의 결과가 Takahashi보다 약 20배 더 약하게 나타났다. Takahashi는 red와 blue lobe를 각각 2개씩 관측했다. Aso는 이 천체에 대하여 분석을 하지 않았다.  SED의 기울기 $\alpha$는 1.51으로 Class I으로 분류하였다. Furlan에서도 Class I으로 분류하였다.\cite{HerschelFurlan} 

FIR6b의 contour map을 살펴보면 주변의 다른 별들로 인해서 방출류 구조 말고 다른 별들에서 나온 방출류로 인한 선들이 많이 보인다. NW-SE 방향으로 방출류가 관측이 된다. 본 연구의 결과가 Takahashi보다 약 4배 더 약하게 나타났다. Aso는 이 천체에 대하여 분석을 하지 않았다. SED의 기울기 $\alpha$는 1.20으로 Class I으로 분류하였다. Furlan에서는 Class 0으로 분류하였다.\cite{HerschelFurlan} 

MMS2의 contour map을 살펴보면 red lobe와 blue lobe가 둘 다 별을 기준으로 동쪽에 있는 특이한 모양을 하고 있다. SW 방향에 보이는 방출류 구조는 MMS5로 인한 방출류이다. 아마 이것에 의해 방출류가 영향을 받아 치우쳐졌을 가능성이 있다. 본 연구의 결과가 Takahashi보다 약 4배 더 약하게 나타났다. J=1-0 을 사용한 Aso보다 약 1.8배 더 강하게 나타났다. Aso는 MMS2, MMS3, MMS4 세 개의 원시성을 하나의 원시성으로 간주하고 방출류를 계산했다.\cite{Aso} SED의 기울기 $\alpha$는 1.23으로 Class I으로 분류하였다. Furlan에서는 Flat으로 분류하였다.\cite{HerschelFurlan} 

MMS5의 contour map을 살펴보면 E-W 방향으로 방출류가 관측이 된다. blue lobe가 red lobe보다 더 강하게 관측된다. 본 연구의 결과가 Takahashi보다 약 4배 더 약하게 나타났다. Aso보다 약 10\% 강하게 나타났다.\cite{Aso} SED의 기울기 $\alpha$가 3.17로 Class I로 분류되어야 하지만 $2.2\mu m$와 $20 \mu m$ 사이의 관측 데이터의 값이 $10^{-15}$ Wm$^{-2}$정도로 매우 작게 나타났기 때문에 Class 0으로 분류하였다. Furlan에서는 Class I으로 분류하였다.\cite{HerschelFurlan} 

MMS9의 contour map을 살펴보면  E-W 방향으로 강한 방출류가 나오는 것을 볼 수 있다. Takahashi(2008)에서는 red lobe를 두개 관측했는데, 본 연구의 관측 자료에서도 blue lobe의 중심 부근에서 작은 red lobe가 존재하는것을 알 수 있다. 그 세기는 main red lobe보다 10배 정도 더 작은것으로 관측되었다. 본 연구의 결과가 Takahashi보다 약 20배 더 약하게 나타났다. Takahashi는 2개의 red lobe를 관측했다.  Aso의 결과에 비해서는 약 6.7배 약하게 나타났다.\cite{Aso} SED의 기울기 $\alpha$는 1.53으로 Class I으로 분류하였다. Furlan에서는 Class 0으로 분류하였다.\cite{HerschelFurlan} 
\\

각 원시성들의 line profile에 나타난 $^{13}CO$와 $C^{18}O$ 천이 선을 보면 red peak, center, blue peak 세 지점에서 모두 비슷한 개형이 나타났다. 따라서 각 원시성 주변의 red, blue lobe가 본 연구에서 관찰한 원시성으로부터 나온 방출류라는 것을 알 수 있다. 그리고 SED로부터 분류한 진화단계가 일부 원시성들은 Furlan의 결과와 다르게 나타났는데, Furlan은 각 원시성들을 bolometric temperature을 기준으로 분류했기 때문에 본 연구에서 SED를 이용해서 구한 classification과 차이가 있을 수 있다.

\clearpage
\newpage   


\subsection{방출류의 세기}
\begin{table}[h]
	\begin{center}
		\begin{tabular}{c|c|c|c}
			\toprule
			\textbf{Name} &$\mathbf{F_{R}}$ & $\mathbf{F_{B}}$ & $\mathbf{F_{CO}}$\\
			& \multicolumn{3}{c}{[M$_{\odot}$ km s$^{-1}$ yr$^{-1}$]}\\
			\midrule
			\multicolumn{4}{c}{Orion A Cloud}\\
			\midrule
			FIR2 & 1.14E-05 & 3.28E-05 & 4.42E-05\\
			FIR3 & 4.77E-03 & 7.43E-03 & 1.22E-04\\
			FIR6b & 1.13E-05 & 1.18E-05 & 2.31E-05\\
			MMS2 & 1.14E-05 & 4.50E-05 & 5.64E-05\\
			MMS5 & 5.80E-06 & 1.55E-05 & 2.13E-05\\
			MMS9 & 3.67E-06 & 1.09E-05 & 1.46E-05\\
			\midrule
			\multicolumn{4}{c}{$\rho$ Ophiuchus Cloud}\\
			\midrule
			Elias 32 & 1.77E-06 & 1.01E-05 & 1.19E-05\\
			IRS 46 & 4.56E-07 & 7.14E-07 & 1.17E-06\\
			VLA 1623 & 2.42E-06 & 3.15E-06 & 5.57E-06\\
			BBRCG 24 & 3.78E-07 & 8.19E-07 & 1.20E-06\\
		\end{tabular}
	\end{center}
	\caption{관측된 원시성들의 방출류의 세기}
\end{table}

표에서 $F_R$와 $F_B$는 각각 red, blue lobe의 방출류의 세기를 구한것이다.(\ref{FCO}) $F_{CO}$는 두 값을 더한 값으로 원시성이 방출해내는 총 방출류의 세기이다. 두 영역의 방출류를 비교해 보면 $\rho$ Ophiuchus Cloud은 Orion A Cloud보다 질량이 작고 광도가 낮은 별들이 탄생하는 영역으로 방출류의 세기 또한 작게 나타났다.\\
