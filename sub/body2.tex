\section{분석}

\subsection{identification}

test


\section{결과}

이후에 나오는 Countour map에서 Orion A Cloud의 경우 10'가 실제 길이로는 1.25pc, $\rho$ Ophiuchus Cloud에서는 10'가 실제 길이로는 0.40pc이다. 그리고 Line profile에서 검은선은 $^{12}CO$ 천이 선, 초록선은 $^{13}CO$ 천이 선, 갈색선은 $C^{18}O$ 천이 선의 line이며 파랑색과 빨강색 점선은 blue, red lobe의 속도 범위를 나타낸 것이다.


\subsection{Orion A Cloud}


FIR2의 contour map을 살펴보면 N-S 방향으로 강한 방출류가 보인다. 방출류의 크기가 약 30 arcsec정도로 다른 방출류들보다 훨씬 작다. 본 연구의 결과가 Takahashi보다 약 3배 더 약하게 나타났다. Aso는 이 천체에 대하여 분석을 하지 않았다.
각 원시성들의 line profile에 나타난 $^{13}CO$와 $C^{18}O$ 천이 선을 보면 red peak, center, blue peak 세 지점에서 모두 비슷한 개형이 나타났다. 따라서 각 원시성 주변의 red, blue lobe가 본 연구에서 관찰한 원시성으로부터 나온 방출류라는 것을 알 수 있다. 그리고 SED로부터 분류한 진화단계가 일부 원시성들은 Furlan의 결과와 다르게 나타났는데, Furlan은 각 원시성들을 bolometric temperature을 기준으로 분류했기 때문에 본 연구에서 SED를 이용해서 구한 classification과 차이가 있을 수 있다. \cite{bontemps1996evolution}


\subsection{방출류의 세기}
\begin{table}[h]
	\begin{center}
		\begin{tabular}{c|c|c|c}
			\toprule
			\textbf{Name} &$\mathbf{F_{R}}$ & $\mathbf{F_{B}}$ & $\mathbf{F_{CO}}$\\
			& \multicolumn{3}{c}{[M$_{\odot}$ km s$^{-1}$ yr$^{-1}$]}\\
			\midrule
			\multicolumn{4}{c}{Orion A Cloud}\\
			\midrule
			FIR2 & 1.14E-05 & 3.28E-05 & 4.42E-05\\
			FIR3 & 4.77E-03 & 7.43E-03 & 1.22E-04\\
			FIR6b & 1.13E-05 & 1.18E-05 & 2.31E-05\\
			MMS2 & 1.14E-05 & 4.50E-05 & 5.64E-05\\
			MMS5 & 5.80E-06 & 1.55E-05 & 2.13E-05\\
			MMS9 & 3.67E-06 & 1.09E-05 & 1.46E-05\\
			\midrule
			\multicolumn{4}{c}{$\rho$ Ophiuchus Cloud}\\
			\midrule
			Elias 32 & 1.77E-06 & 1.01E-05 & 1.19E-05\\
			IRS 46 & 4.56E-07 & 7.14E-07 & 1.17E-06\\
			VLA 1623 & 2.42E-06 & 3.15E-06 & 5.57E-06\\
			BBRCG 24 & 3.78E-07 & 8.19E-07 & 1.20E-06\\
		\end{tabular}
	\end{center}
	\caption{관측된 원시성들의 방출류의 세기}
\end{table}

 Ophiuchus Cloud은 Orion A Cloud보다 질량이 작고 광도가 낮은 별들이 탄생하는 영역으로 방출류의 세기 또한 작게 나타났다. \cite{megeath2012spitzer}\\
