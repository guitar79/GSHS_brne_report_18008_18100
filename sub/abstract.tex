%\maketitle  % command to print the title page with above variables
\makecover  % command to print the title page with above variables

\setcounter{page}{1}
\renewcommand{\thepage}{\roman{page}}

%----------------------------------------------
%   Table of Contents (자동 작성됨)
%----------------------------------------------
\cleardoublepage
\addcontentsline{toc}{section}{Contents}
\setcounter{secnumdepth}{3} % organisational level that receives a numbers
\setcounter{tocdepth}{3}    % print table of contents for level 3
\baselineskip=2.2em
\tableofcontents


%----------------------------------------------
%     List of Figures/Tables (자동 작성됨)
%----------------------------------------------
\cleardoublepage
\clearpage
\listoffigures	% 그림 목록과 캡션을 출력한다. 만약 논문에 그림이 없다면 이 줄의 맨 앞에 %기호를 넣어서 코멘트 처리한다.

\cleardoublepage
\clearpage
\listoftables  % 표 목록과 캡션을 출력한다. 만약 논문에 표가 없다면 이 줄의 맨 앞에 %기호를 넣어서 코멘트 처리한다.


\cleardoublepage
\clearpage

%---------------------------------------------------------------------
%                  영문 초록을 입력하시오
%---------------------------------------------------------------------
\begin{abstracts}     %this creates the heading for the abstract page
	\addcontentsline{toc}{section}{Abstract}  %%% TOC에 표시
	\noindent{
			Put your abstract here. Once upon a time, \gshs said : `The first, and the best.'
	}
\end{abstracts}

\cleardoublepage
\clearpage

\begin{abstractskor}
	\addcontentsline{toc}{section}{초록}  %%% TOC에 표시
	\noindent{
우리가 천체를 관측하기 위해서 천체망원경을 자주 사용하고는 한다. 하지만 천체망원경으로 초점을 정확하게 맞춰야지만 보다 정확한 천체관측을 진행할 수 있다. 대부분 망원경에는 이미 초점을 맞추기 위해 손으로 직접 접안렌즈의 거리를 조절하거나, 조절할 수 있는 모터를 이용해 정확하게 우리 눈으로 초점을 맞출 수 있게 만들어져 있다. 하지만, 모터를 이용하여 초점을 맞춘다고 해도 우리 눈으로 초점을 맞추는 것이기 때문에 정확하지 않을 수 있다. 본 논문에서는 아두이노를 이용하여 모터를 돌려 초점이 맞춰졌는지 관측하기 위한 구체적인 방안을 제시한다. 이뿐만이 아니라, 이를 이용하여 센서를 이용하여 자동으로 초점을 맞추는 기계를 만드는 것에 대회여 탐구하였다. 만약 이를 실현하게 한다면 천체망원경을 이용하여 여러 천체를 관측하는 데 큰 도움이 될 수 있을 것이다.
	}
\end{abstractskor}






